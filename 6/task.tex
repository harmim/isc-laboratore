\documentclass{article}
\usepackage[utf8]{inputenc}
\usepackage[czech]{babel}
\title{Quicksort}
\author{Vojtěch Bartl}
\date{\today}

\begin{document}
\maketitle
\section{Úvod}
Quicksort (česky „rychlé řazení“) je jeden z nejrychlejších běžných algoritmů řazení založených na porovnávání prvků. Jeho průměrná časová složitost je pro algoritmy této skupiny nejlepší možná ($O(N\,log\,N)$), v nejhorším případě (kterému se ale v praxi jde obvykle vyhnout) je však jeho časová náročnost $O(N^2)$. Další výhodou algoritmu je jeho jednoduchost.

Objevil jej Sir Charles Antony Richard Hoare v roce 1961.
\section{Algoritmus}

Základní myšlenkou quicksortu je rozdělení řazené posloupnosti čísel na dvě přibližně stejné části (quicksort patří mezi algoritmy typu rozděl a panuj). V jedné části jsou čísla větší a ve druhé menší, než nějaká zvolená hodnota (nazývaná pivot – anglicky „střed otáčení“). Pokud je tato hodnota zvolena dobře, jsou obě části přibližně stejně velké. Pokud budou obě části samostatně seřazeny, je seřazené i celé pole. Obě části se pak rekurzivně řadí stejným postupem, což ale neznamená, že implementace musí taky použít rekurzi.
Volba pivotu

\subsection{Volba Pivotu}
Největším problémem celého algoritmu je volba pivotu. Pokud se daří volit číslo blízké mediánu řazené části pole, je algoritmus skutečně velmi rychlý. V opačném případě se jeho doba běhu prodlužuje a v extrémním případě je časová složitost $O(N^2)$. Přirozenou metodou na získání pivotu se pak jeví volit za pivot medián. Hledání mediánu (a obecně k-tého prvku) v posloupnosti běží v lineárním čase vzhledem k počtu prvků, tím dostaneme složitost $O(N\,log\,N)$ quicksortu v nejhorším případě. Nicméně tato implementace není příliš rychlá z důvodu vysokých konstant schovaných v O notaci. Proto existuje velké množství alternativních způsobů, které se snaží efektivně vybrat pivot co nejbližší mediánu. Zde je seznam některých metod:
\begin{itemize}
    \item{První prvek – popřípadě kterákoli jiná fixní pozice. (Fixní volba prvního prvku je velmi nevýhodná na částečně seřazených množinách.)}
    \item {Náhodný prvek – často používaná metoda. Průměr přes každá data je $O(N\,log\,N)$, přičemž zde se průměr bere přes všechny možné volby pivotů (rozděleno rovnoměrně). Nejhorší případ zůstává $O(N^2)$, protože pro každá data může náhoda nebo Velmi Inteligentní Protivník vybírat soustavně nevhodného pivota, např. druhé největší číslo. V praxi většinou není dostupný generátor skutečně náhodných čísel, proto se používá pseudonáhodný výběr.}
    \item{Metoda mediánu tří – případně pěti či jiného počtu prvků. Pomocí pseudonáhodného algoritmu (také se používají fixní pozice, typicky první, prostřední a poslední) se vybere několik prvků z množiny, ze kterých se vybere medián, a ten je použit jako pivot.}
\end{itemize}
Pokud by bylo zaručeno, že pivota volíme vždy z $98\,\%$ prvků uprostřed a ne z $1\,\%$ na některé straně, algoritmus by stále měl nejhorší asymptotickou složitost $O(N\,log\,N)$, byť s poněkud větší konstantou v $O$-notaci.

Praktické zkušenosti a testy ukazují, že na pseudonáhodných nebo reálných datech je Quicksort nejrychlejší ze všech obecných řadicích algoritmů (tedy i rychlejší než Heapsort a Mergesort, které jsou formálně rychlejší). Rychlost Quicksortu však není zaručena pro všechny vstupy. Maximální časová náročnost $O(N^2)$ Quicksort diskvalifikují pro kritické aplikace.

\end{document}